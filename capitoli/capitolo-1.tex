% !TEX encoding = UTF-8
% !TEX TS-program = pdflatex
% !TEX root = ../tesi.tex

%**************************************************************
\chapter{Introduzione}
\label{cap:introduzione}
%**************************************************************

\section{Scopo del documento}

Questo documento è la relazione conclusiva del tirocinio frequentato dallo
studente in conclusione al piano di studio. \\
Si cerca qui di spiegare il lavoro svolto mostrandone i risultati, il percorso, 
e le relative scelte tecnologiche/architetturali, oltre che le competenze
coinvolte per tale raggiungimento.

%**************************************************************
\section{L'azienda}
Sync Lab è un azienda che realizza prodotti e soluzioni per diversi amibiti come
sanità, industria, energia, telcomunicazioni, finanza, trasporti e logistica.
Si è specializzata in campi come GDPR, Big Data, Cloud Computing, IoT, Mobile e Cyber Security.
Un azienda con una storia di vent' anni e sedi a Napoli, Roma, Padova, Verona, Milano, Como;
un organico di più di 300 persone, un fatturato di circa 13 milioni di euro e numerose certificazioni
ISO in campi di qualità, gestione dati, sicurezza, gestione ambientale.
Tra i suoi più di 150 clienti troviamo alcuni come TIM, Vodafone, Intesa San Paolo, Enel e Trentalia.

%**************************************************************
\section{Struttura del testo}

{\Huge TODO}
\begin{description}
    \item[{\hyperref[cap:descrizione-stage]{Il secondo capitolo}}] approfondisce ...
    
    \item[{\hyperref[cap:analisi-requisiti]{Il terzo capitolo}}] approfondisce ...
    
    \item[{\hyperref[cap:progettazione-codifica]{Il quarto capitolo}}] approfondisce ...
    
    \item[{\hyperref[cap:verifica-validazione]{Il quinto capitolo}}] approfondisce ...
    
    \item[{\hyperref[cap:conclusioni]{Nel sesto capitolo}}] descrive ...
\end{description}

\section{Convenzioni tipografiche}
Riguardo la stesura del testo, relativamente al documento sono state adottate le seguenti convenzioni tipografiche:
\begin{itemize}
	\item Acronimi, le abbreviazioni e i termini ambigui o di uso non comune menzionati vengono
        definiti nel glossario, situato alla fine di questo documento;
    \item La prima occorrenza dei termini riportati nel glossario viene denotata da una $'g'$ in
        apice alla parola: \gls{umlg};
	\item Termini in lingua straniera o facenti parti del gergo tecnico sono evidenziati con il
        carattere \emph{corsivo}.
	\item  Snippet di codice, nomi di file e simili verranno indicati con il \texttt{carattere monospace};
\end{itemize}